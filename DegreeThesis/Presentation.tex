\documentclass{beamer}
\usepackage[utf8]{inputenc}
\usepackage[russian]{babel}
\usepackage{graphicx}
\usepackage{movie15}
\usepackage{fontawesome5}
\usepackage{tikz}
\usepackage{adjustbox}
\usepackage[most]{tcolorbox}
\usepackage{lmodern}
\usepackage{enumitem}

% Современная тема и настройки
\usetheme{metropolis}
\setbeamercolor{background canvas}{bg=white}
\setbeamercolor{frametitle}{bg=black,fg=white}
\setbeamercolor{progress bar}{fg=green!40!black,bg=gray!30}
\setbeamertemplate{frame numbering}[fraction]
\setbeamertemplate{navigation symbols}{}

% Упрощенная версия progress bar
\makeatletter
\def\progressbar@progressbar{%
    \begin{tikzpicture}
        \draw[progress bar.bg] (0,0) rectangle (\paperwidth,0.4pt);
        \draw[progress bar.fg] (0,0) rectangle (\insertframenumber\paperwidth/\inserttotalframenumber,0.4pt);
    \end{tikzpicture}%
}

% Добавляем progress bar в footer
\addtobeamertemplate{footline}{}{%
    \begin{beamercolorbox}[wd=\paperwidth,ht=1pt,dp=0pt]{progress bar}
        \progressbar@progressbar
    \end{beamercolorbox}%
}
\makeatother

% Добавляем новые цвета и настройки
\definecolor{accent}{RGB}{41,128,185}
\definecolor{light-accent}{RGB}{52,152,219}
\definecolor{emphasis}{RGB}{231,76,60}

% Настраиваем tcolorbox
\tcbset{
    boxrule=0pt,
    frame hidden,
    borderline west={3pt}{0pt}{accent},
    colback=black!3,
    sharp corners,
    title style={sharp corners, colback=accent!10},
    fonttitle=\bfseries
}

% Определяем стиль для важных блоков
\newtcolorbox{highlight}{
    colback=accent!5,
    boxrule=0pt,
    frame hidden,
    borderline={3pt}{0pt}{accent},
    sharp corners
}

\begin{document}

\begin{frame}[plain,c]
  \centering
  \vspace{1em}

  % Университет
  {\large Новосибирский государственный университет, НГУ\par}
  \vspace{0.5em}

  % Факультет/специальность
  {\normalsize 09.03.01 Информатика и вычислительная техника\\
   Программная инженерия и компьютерные науки\par}
  \vspace{1.2em}

  % Название работы – теперь жирное и более заметное
  {\Large Разработка сервиса выделения\\
   видеофрагментов по текстовой информации\par}
  \vspace{1.8em}

  \begin{flushleft}
	  {\small Студент: Ланин Даниил Михайлович\par}
	  {\small Руководитель ВКР: Яхъяева Гульнара Эркиновна,\\
	   к.ф.-м.н., доцент кафедры общей информатики ФИТ\par}
	  {\small Соруководитель ВКР: Худяков Даниил Александрович,\\
	   ассистент кафедры общей информатики ФИТ\par}
  \end{flushleft}

  \vspace{1.0em}
  % Дата
  {\normalsize 17.06.2025\par}
  \vspace{1em}
\end{frame}


\begin{frame}{Актуальность}
    \begin{itemize}
        \item \faChartLine\ Экспоненциальный рост данных в интернете
        \item \faVideo\ Видео — доминирующий формат интернет-трафика
        \item \faSearch\ Сложность поиска информации в видео
        \item \faClock\ 500+ часов видео загружается на YouTube каждую минуту
        \item \faExclamationTriangle\ Проблема эффективного поиска становится критической
    \end{itemize}
\end{frame}

\begin{frame}{Рост видеоконтента}
    \centering
    \includegraphics[width=0.75\textwidth]{media/video_platform_growth.jpg}
    \begin{tcolorbox}[colback=black!5,colframe=black!20,title=www.statista.com]
        К началу 2025 года около 76\% всего мобильного трафика генерируется видеоприложениями!
    \end{tcolorbox}
\end{frame}

\begin{frame}{Цель и задачи проекта}
    \small Цель: Создание сервиса, позволяющего автоматически выделять релевантные фрагменты видео по текстовому запросу.

    % Два столбца с задачами и результатами
    \begin{columns}[T]
        \begin{column}{0.52\textwidth}
            \begin{tcolorbox}[title=\faTasks\ Основные задачи,
                              left=5pt,
                              right=5pt]
                \begin{itemize}[nosep, leftmargin=*]
                    \item \faServer\ Создание асинхронного веб-приложения
                    \item \faSearch\ Реализация умного поиска по видео
                    \item \faHighlighter\ Подсветка релевантных фрагментов
                    \item \faLanguage\ Поддержка множества языков
                    \item \faDownload\ Сохранение найденных фрагментов
                    \item \faChartLine\ Оптимизация производительности
                \end{itemize}
            \end{tcolorbox}
        \end{column}
        \begin{column}{0.48\textwidth}
            \begin{tcolorbox}[title=\faLightbulb\ Ожидаемые результаты,
                              left=5pt,
                              right=5pt]
                \begin{itemize}[nosep, leftmargin=*]
                    \item Быстрый поиск
                    \item Точные результаты
                    \item Удобный интерфейс
                \end{itemize}
            \end{tcolorbox}
        \end{column}
    \end{columns}
\end{frame}

\begin{frame}{Области применения}
    \begin{columns}[T]
        \begin{column}{0.5\textwidth}
            \begin{tcolorbox}[title=\faGraduationCap\ Образование,
                            left=5pt,
                            right=5pt]
                \begin{itemize}[nosep, leftmargin=*]
                    \item Поиск по видеолекциям
                    \item Создание учебных материалов
                    \item Анализ контента
                \end{itemize}
            \end{tcolorbox}
            \vspace{0.3cm}
            \begin{tcolorbox}[title=\faLanguage\ Изучение языков,
                            left=5pt,
                            right=5pt]
                \begin{itemize}[nosep, leftmargin=*]
                    \item Поиск примеров произношения
                    \item Изучение разговорной речи
                    \item Анализ диалектов
                \end{itemize}
            \end{tcolorbox}
        \end{column}
        \begin{column}{0.5\textwidth}
            \begin{tcolorbox}[title=\faSearchPlus\ Исследования,
                            left=5pt,
                            right=5pt]
                \begin{itemize}[nosep, leftmargin=*]
                    \item Анализ интервью
                    \item Обработка данных
                    \item Создание баз знаний
                \end{itemize}
            \end{tcolorbox}
            \vspace{0.3cm}
            \begin{tcolorbox}[title=\faBuilding\ Бизнес,
                            left=5pt,
                            right=5pt]
                \begin{itemize}[nosep, leftmargin=*]
                    \item Анализ совещаний
                    \item Обучение персонала
                    \item Архивация данных
                \end{itemize}
            \end{tcolorbox}
        \end{column}
    \end{columns}
\end{frame}

\begin{frame}{Технологии и инструменты разработки}
    \begin{columns}[T]
        \begin{column}{0.48\textwidth}
            \begin{tcolorbox}[title=\faRobot\ Машинное обучение,
                            left=5pt,
                            right=5pt]
                \begin{itemize}[nosep, leftmargin=*]
                    \item \textcolor{accent}{Whisper} для транскрибции
                \end{itemize}
            \end{tcolorbox}
            \vspace{0.3cm}
            \begin{tcolorbox}[title=\faChartLine\ Оптимизация,
                            left=5pt,
                            right=5pt]
                \begin{itemize}[nosep, leftmargin=*]
                    \item Кэширование запросов
                    \item Параллельное выполнение запросов
                    \item 
                \end{itemize}
            \end{tcolorbox}
        \end{column}
        \begin{column}{0.48\textwidth}
            \begin{tcolorbox}[title=\faServer\ Архитектура,
                            left=5pt,
                            right=5pt]
                \begin{itemize}[nosep, leftmargin=*]
                    \item \textcolor{accent}{FastAPI} backend
                    \item \textcolor{accent}{Celery} для асинхронности
                    \item \textcolor{accent}{Elasticsearch} для поиска
                \end{itemize}
            \end{tcolorbox}
            \vspace{0.3cm}
            \includegraphics[width=\linewidth]{media/ml_technology.png}
        \end{column}
    \end{columns}
\end{frame}

\begin{frame}{Архитектура приложения}
    \begin{itemize}
        \item Асинхронная архитектура на основе FastAPI и Celery.
        \item Высокая производительность и масштабируемость.
        \item Параллельная обработка множества запросов.
        \item Минимизация времени отклика.
    \end{itemize}
    \vfill
    \centering
    \begin{minipage}{0.6\textwidth}
        \includegraphics[width=\linewidth]{media/celery.png} % celery
    \end{minipage}
    \hfill
    \begin{minipage}{0.35\textwidth}
        \includegraphics[width=\linewidth]{media/tools.png} % tools
    \end{minipage}
\end{frame}

\begin{frame}{Обработка видео}
  \begin{itemize}
    \item \faUpload\ Загрузка видео на сервер
    \item \faClosedCaptioning\ Генерация субтитров с помощью \textcolor{accent}{Whisper}
    \item \faSearch\ Индексация субтитров в \textcolor{accent}{Elasticsearch}
    \item \faDatabase\ Сохранение всех данных в БД
  \end{itemize}
  \vfill
  \hspace*{-0.09\linewidth}%
  \includegraphics[width=1.2\linewidth]{media/processing_video.png}
\end{frame}

\begin{frame}{Поиск контента}
  \begin{itemize}
    \item \faDatabase\ Использование \textcolor{accent}{Elasticsearch}
    \item \faSearchPlus\ Поддержка точного и неточного поиска:
      \begin{itemize}[nosep]
        \item \faBullseye\ \textbf{Точный поиск} — строгое соответствие запросу
        \item \faMagic\ \textbf{Неточный поиск} — совпадение слов/выражений по схожести (Эмбеддинг)
      \end{itemize}
  \end{itemize}
  \vfill
  \centering
  \includegraphics[width=\linewidth]{media/search_video.png}
\end{frame}

\begin{frame}{Интерфейс пользователя}
    \begin{itemize}[label={}, leftmargin=*]
        \item \faSearch\ Интуитивно понятный интерфейс
        \item \faFilter\ Возможность просматривания загруженных видео с быстрым поиском фрагментов по этому видео
        \item \faHighlighter\ Подсветка ключевых слов
        \item \faDownload\ Сохранение найденных фрагментов
        \item \faHistory\ Просмотр истории поиска
    \end{itemize}
\end{frame}

\begin{frame}{Демонстрация интерфейса - Поиск}
    \centering
    \includegraphics[width=0.85\textwidth]{media/web_interface_1.png}
    \vfill
    \begin{tcolorbox}[title=\faSearch\ Поисковый интерфейс]
        Интеллектуальная система поиска с историей запросов
    \end{tcolorbox}
\end{frame}

\begin{frame}{Демонстрация интерфейса - Результаты}
    \centering
    \includegraphics[width=0.85\textwidth]{media/web_interface_2.png}
    \vfill
    \begin{tcolorbox}[title=\faPlayCircle\ Просмотр результатов]
        Удобный просмотр найденных фрагментов с подсветкой
    \end{tcolorbox}
\end{frame}

\begin{frame}{Демонстрация работы системы}
	% тут надо включить видео
\end{frame}

\begin{frame}{Преимущества проекта}
    \begin{itemize}[nosep]
        \item \faRocket\ Универсальная асинхронная архитектура (Использование Celery и FastAPI)
        \item \faUpload\ Возможность загрузки своего видеоролика с автоматической генерацией субтитров
        \item \faLanguage\ Нейронная сеть поддерживает 99 различных языков
        \item \faSave\ Возможность сохранить найденный видеофрагмент
        \item \faSearch\ Поддержка неточного поиска по схожесть (Эмбеддинг и кластеризация)
    \end{itemize}
\end{frame}

\begin{frame}{Результаты проделанной работы}
    \begin{highlight}
        \textbf{\faCheckCircle\ Основные результаты:}
        \begin{itemize}
            \item \textcolor{emphasis}{Разработан} функциональный прототип сервиса
            \item \textcolor{emphasis}{Реализован} эффективный поиск по видео
            \item \textcolor{emphasis}{Внедрены} современные ML-технологии
            \item \textcolor{emphasis}{Оптимизирована} обработка видеоданных
        \end{itemize}
    \end{highlight}
\end{frame}

\begin{frame}{Публикации}
    \small
    \begin{enumerate}[label=\arabic*. , leftmargin=*]
        \item Ланин Д. М. Разработка сервиса выделения видеофрагмента по текстовой информации. \\
              Тезисы докладов Международной конференции «Мальцевские чтения». Новосибирск, 2024. С. 64.
        \item Ланин Д. М. Разработка сервиса выделения видеофрагментов по текстовой информации. \\
              Тезисы докладов 63-й Международной научной студенческой конференции. Новосибирск, 2025. Принято к печати.
    \end{enumerate}
\end{frame}

\setbeamertemplate{navigation symbols}{}
\setbeamertemplate{footline}[frame number]

\end{document}
